
% ToDo:
% extra section level
% design: table
% DinPro & Inconsolata: slanted, small-caps
% wrapfig richtig einrichten (ohne etoolbox)
% fußnoten hyperlinks
% komischer header bug
%



%----------------- Meta -----------------------
% Titel, Author, ...
% 
% 
% 
% 
\begin{filecontents*}[overwrite]{\jobname.xmpdata}
    \Title{16F84-Simulator}
    \Author{Valerio Cocco \& Daniel Ritterhofer}
    \Language{de}
    \Keywords{16F84-Simulator, Emulator, Projekt}
\end{filecontents*}
% 
% 
% 
% 
% 
% 
% 
% 
% 
% 
% 
% 
% 
% 
% 
% 
% 
% 
%  % Metadaten: pdfa.xmpi
%----------------- Meta -----------------------



\PassOptionsToPackage{hyperfootnotes=false}{hyperref}



\documentclass[
  ngerman, % neue deutsche Rechtschreibung
  a4paper, % Papiergrösse
  % twoside, % Zweiseitiger Druck (rechts/links)
  12pt, % Schriftgrösse
  parskip=full-,
  % parskip=half,
]{scrartcl} % “article”-like class of the koma-script collection - https://ctan.org/pkg/scrartcl



%----------------- packages ---------------------------------------------------------------------------------------------------
\usepackage[a-3b]{pdfx} % PDF/X and PDF/A support - https://ctan.org/pkg/pdfx
\usepackage{babel} % Sprachanpassungen für generierte Texte wie "Inhaltsverzeichnis" etc - https://ctan.org/pkg/babel
\usepackage[babel, autostyle, german=guillemets]{csquotes} % Context sensitive quotation facilities - https://ctan.org/pkg/csquotes
\usepackage{xspace} % space commands - https://ctan.org/pkg/xspace
\usepackage{bookmark} % Bookmarks for hyperref - https://ctan.org/pkg/bookmark
\usepackage{graphicx} % Enhanced support for graphics - https://ctan.org/pkg/graphicx
\usepackage{pdfpages} % Include PDF documents - https://ctan.org/pkg/pdfpages
\usepackage{framed} % framed environemnt and more - https://ctan.org/pkg/framed
\usepackage{setspace} % Abstand zwischen den Zeilen. \onehalfspacing % https://ctan.org/pkg/setspace
\usepackage{float} % Improved interface for floating objects - https://ctan.org/pkg/float
\usepackage{wrapfig} % Produces figures which text can flow around - https://www.ctan.org/pkg/wrapfig
\usepackage{caption} % Customising captions in floating environments - https://www.ctan.org/pkg/caption
%----------------- packages ---------------------------------------------------------------------------------------------------



%-------------------- header & footer ---------------------------------------------------------------------
\usepackage{scrlayer-scrpage} % Define and manage page styles - https://ctan.org/pkg/scrlayer-scrpage
\clearpairofpagestyles
\automark[section]{section} % verstehe ich nicht: vermutlich wegen package clash nötig :/
\ihead{16F84-Simulator}
\chead{}
\ohead{\rightmark} % section titel links
\ifoot{© Daniel Rittershofer \& Valerio Cocco}
\ofoot{\pagemark}
\addtokomafont{pagehead}{\footnotesize\upshape\sffamily}
\addtokomafont{pagefoot}{\footnotesize\upshape\sffamily}
\addtokomafont{pagenumber}{\bfseries\footnotesize\sffamily}
%-------------------- header & footer ---------------------------------------------------------------------



% Seitenlayout
\usepackage[
    a4paper,
    top=2.5cm,
    headsep=1cm,
    left=1.8cm,
    right=1.8cm,
    bottom=2.2cm,
    footskip=1.2cm,
]{geometry} % interface to document dimensions - https://ctan.org/pkg/geometry



%------------------ Farben ----------------------------------------------------------
\usepackage{xcolor} % Extending LATEX color facilities % https://ctan.org/pkg/xcolor
\definecolor{Mittelblau}{RGB}{55,74,154}
\definecolor{Horizontorange}{RGB}{255,153,0}
\definecolor{Grau50}{RGB}{151,151,151}
\definecolor{Grau70}{RGB}{104,104,104}
\definecolor{Grau90}{RGB}{85,85,85}
\definecolor{Anthrazitgrau}{RGB}{59,59,59}
\definecolor{Gelb}{RGB}{246,194,81}
\definecolor{solarized-base-2}{RGB}{252,251,247}
\definecolor{solarized-base-3}{RGB}{253,246,227}
%------------------ Farben ----------------------------------------------------------



%-------------------------------------------------- Schrift -------------------------------------------------------------------
\usepackage{fontspec} % Schrift aus Dateien laden. https://ctan.org/pkg/fontspec

\usepackage{libertineRoman}

\setmonofont{JetBrainsMono}[
  Path = ./fonts/JetBrainsMono/,
  Extension = .otf,
  % 
  UprightFont = *-Light,
  BoldFont = *-Bold,
  ItalicFont = *-LightItalic,
  BoldItalicFont = *-BoldItalic,
  Scale=0.875,
]

\usepackage{microtype} % Subliminal refinements towards typographical perfection - https://ctan.org/pkg/microtype
%-------------------------------------------------- Schrift -------------------------------------------------------------------



%-------------------- Bibliographie ------------------------------------------------------------------
\usepackage[
	hyperref=true,          % Klickbare Referenzen in der PDF-Datei
  backref=true,           % In der Literaturref. die Seiten angeben, wo ein \cite dazu steht
	style=ext-authoryear-comp,
  maxnames=1,
  backend=biber,
  sortlocale=auto
	sorting=nty]{biblatex} % https://ctan.org/pkg/biblatex

% Blockzitate csquote
\renewcommand{\mkbibnamefamily}[1]{\textsc{#1}}
\DeclareFieldFormat{postnote}{\textsl{#1}}

% Abstand der einzelnen Einträge in der Literaturangaben
\renewcommand{\bibitemsep}{1ex}

% Welche Klammern soll \parencite{..} benutzen? csquote
\DeclareOuterCiteDelims{parencite}{\bibopenbracket}{\bibclosebracket}

% biblatex's \cite{..} gibt "normalerweise" keine Klammern um die Referenz aus
% \parencite{..} gibt Klammern aus (die oben definiert sind)
% Damit das "normale" Verhalten andere BibTeX-Stile realisiert wird, d.h. \cite{..}
% gibt Klammen aus, wird folgendes definiert:
% "Merke" ursprünliche Definition unter neuem Namen
\let\citeNoParen\cite
% "Redefiniere" \cite:
\let\cite\parencite

\let\citesNoParen\cites
% "Redefiniere" \cites:
\let\cites\parencites

\renewcommand{\mkccitation}[1]{ #1}
\renewcommand{\mkcitation}[1]{ #1}
\renewcommand{\mkbegdispquote}[2]{\itshape}{}
\renewcommand{\mkenddispquote}[2]{\normalfont#2}{}


% \bibliographystyle{alphadin}
\bibliography{literatur} % literatur.bib laden
%-------------------- Bibliographie ------------------------------------------------------------------



%--------------------------- listing -----------------------------------------------------
% \usepackage{listings} % Typeset source code listings - https://ctan.org/pkg/listings
\usepackage[most, minted]{tcolorbox}
% \usepackage[most, listings]{tcolorbox}

% Wie sollen die Überschriften benannt werden:
\renewcommand{\lstlistingname}{Auflistung}

\renewcommand{\lstlistlistingname}{Liste der Auflistungen}

% \lstset{
%   numbers=left,              % Zeilennummern einfügen
%   numbersep=10pt,             % Abstand der Nummern zum Text
%   basicstyle=\ttfamily\small,         % Wie soll der Code gesetzt werden
%   aboveskip=\intextsep,
%   belowcaptionskip=\bigskipamount,
%   frame=single,           % Ein Rahmen um den Code
%   framexleftmargin=1.5pt,  % Rahmen link von den Zahlen
%   framexrightmargin=25pt,
%   backgroundcolor = \color{Gelb!10},
%   rulecolor = \color{Mittelblau},
%   numberstyle=\sffamily,
%   breaklines=true,
%   framexbottommargin=5pt,
% }
  
% custom syntax highlighting definitions 
% enabled: false
%
%
%
% \lstdefinelanguage{docker}{
%   keywords={FROM, RUN, COPY, ADD, ENTRYPOINT, CMD,  ENV, ARG, WORKDIR, EXPOSE, LABEL, USER, VOLUME, STOPSIGNAL, ONBUILD, MAINTAINER},
%   keywordstyle=\color{blue}\bfseries,
%   identifierstyle=\color{black},
%   sensitive=false,
%   comment=[l]{\#},
%   commentstyle=\color{purple}\ttfamily,
%   stringstyle=\color{red}\ttfamily,
%   morestring=[b]',
%   morestring=[b]"
% }

% \lstdefinelanguage{docker-compose}{
%   keywords={image, environment, ports, container_name, ports, volumes, links},
%   keywordstyle=\color{blue}\bfseries,
%   identifierstyle=\color{black},
%   sensitive=false,
%   comment=[l]{\#},
%   commentstyle=\color{purple}\ttfamily,
%   stringstyle=\color{red}\ttfamily,
%   morestring=[b]',
%   morestring=[b]"
% }
% \lstdefinelanguage{docker-compose-2}{
%   keywords={version, volumes, services},
%   keywordstyle=\color{blue}\bfseries,
%   keywords=[2]{image, environment, ports, container_name, ports, links, build},
%   keywordstyle=[2]\color{olive}\bfseries,
%   identifierstyle=\color{black},
%   sensitive=false,
%   comment=[l]{\#},
%   commentstyle=\color{purple}\ttfamily,
%   stringstyle=\color{red}\ttfamily,
%   morestring=[b]',
%   morestring=[b]"
% }

% \lstset{basicstyle=\ttfamily,
%   showstringspaces=false,
%   commentstyle=\color{red},
%   keywordstyle=\color{blue},
%   inputencoding=utf8,
%   extendedchars=true
% }

% \newcommand\YAMLcolonstyle{\color{red}\mdseries}
% \newcommand\YAMLkeystyle{\color{black}\bfseries\ttfamily}
% \newcommand\YAMLvaluestyle{\color{blue}\mdseries}

% \makeatletter

% % here is a macro expanding to the name of the language
% % (handy if you decide to change it further down the road)
% \newcommand\language@yaml{yaml}

% \expandafter\expandafter\expandafter\lstdefinelanguage
% \expandafter{\language@yaml}
% {
%   keywords={true,false,null,y,n},
%   keywordstyle=\color{darkgray}\bfserie,
%   basicstyle=\YAMLkeystyle,                                 % assuming a key comes first
%   sensitive=false,
%   comment=[l]{\#},
%   morecomment=[s]{/*}{*/},
%   commentstyle=\color{purple},
%   stringstyle=\YAMLvaluestyle,
%   moredelim=[l][\color{orange}]{\&},
%   moredelim=[l][\color{magenta}]{*},
%   moredelim=**[il][\YAMLcolonstyle{:}\YAMLvaluestyle]{:},   % switch to value style at :
%   morestring=[b]',
%   morestring=[b]",
%   literate =    {---}{{\ProcessThreeDashes}}3
%                 {>}{{\textcolor{red}\textgreater}}1     
%                 {|}{{\textcolor{red}\textbar}}1 
%                 {\ -\ }{{\mdseries\ -\ }}3,
% }

% % switch to key style at EOL
% \lst@AddToHook{EveryLine}{\ifx\lst@language\language@yaml\YAMLkeystyle\fi}
% \makeatother

\usemintedstyle{xcode}
\renewcommand{\theFancyVerbLine}{\tiny\sffamily \textcolor{black}{\arabic{FancyVerbLine}}}
\newtcbinputlisting{js}[2][]{%
  listing engine=minted,
  % listing engine=listings,
  minted language=javascript,
  listing only,
  breakable,
  enhanced,
  minted options = {
    linenos, 
    breaklines=true, 
    breakbefore=., 
    fontsize=\footnotesize, 
    numbersep=2.1mm,
    xleftmargin=0.8mm,
  baselinestretch=1.2
  },
    % listing options={
    %     language=Assembler,
    %     basicstyle=\ttfamily\footnotesize,
    %     numbers=left,
    %     breaklines=true,
    %     numberstyle=\tiny\ttfamily,
    %     xleftmargin=0mm,
    %     numbersep=2.5mm,
    % },
  overlay={%
      \begin{tcbclipinterior}
          \fill[solarized-base-3] (frame.south west) rectangle ([xshift=4.6mm]frame.north west);
      \end{tcbclipinterior}
  },
  colframe=black,
  boxrule=0.8pt,
  colback=solarized-base-2,
  sharp corners,
  top=0pt,
  bottom=0pt,
  % title=#1,
  after={\captionof*{figure}{#1}},
  listing file =#2
}

% 
% 
% 
% 
% 
% 
% 
% 
% 
% 
% 
% 
% 
% 
% 
% 
% 
% 
% 
% 
% 
%--------------------------- listing -----------------------------------------------------



%-------------------------------- Index --------------------------
%\usepackage{makeidx} % Standard package for creating indexes - https://ctan.org/pkg/makeidx
%\makeindex

% \newcommand{\Def}[2][]{%
%    \def\OPTARG{#1}%
%    \def\EMPTY{}%
%    \ifx\OPTARG\EMPTY\index{#2}\else\index{#1}\fi%
%    \textbf{#2}\xspace%
% }
%-------------------------------- Index --------------------------



%-------------------------------- Formelverzeichnis --------------
% \newfloat{eq}{H}{for}[section]
% \newcommand{\forname}{Formelverzeichnis}
% \newcommand{\listofequations}{\listof{eq}{\forname}}

% \newcommand{\eqlabel}[2]{
%     \label{#1}
%     \addcontentsline{for}{eq}{(\ref{#1}) #2}
% }
%-------------------------------- Formelverzeichnis --------------



%------------------------- Links ------------------------------------------------------
\PassOptionsToPackage{hyphens}{url} % allows breaks after explicit hyphen characters
\usepackage{url} % command \url URL-sensitive line breaks - https://ctan.org/pkg/url
\hypersetup{
  % 
  colorlinks=true, % false --> black
  linkcolor=Anthrazitgrau,
  filecolor=Anthrazitgrau,
  citecolor = black,      
  urlcolor=blue,
  % pdfborderstyle={/S/U/W 1},
}
\urlstyle{sf}
%------------------------- Links ------------------------------------------------------



%-------------- reference style ----------------------------------
\newcommand{\reffont}{\sffamily}% References will be \small
\AtBeginDocument{%
  \LetLtxMacro\oldref\ref% Capture \ref in \oldref
  \renewcommand{\ref}[1]{% Update \ref to use...
    {\reffont\oldref{#1}}%
  }%
}
\AtBeginDocument{%
  \LetLtxMacro\oldpageref\pageref% Capture \ref in \oldref
  \renewcommand{\pageref}[1]{% Update \ref to use...
    {\reffont\oldpageref{#1}}%
  }%
}
\AtBeginDocument{%
  \LetLtxMacro\oldnameref\nameref% Capture \ref in \oldref
  \renewcommand{\nameref}[1]{% Update \ref to use...
    {\reffont\oldnameref{#1}}%
  }%
}
\AtBeginDocument{%
  \LetLtxMacro\oldautoref\autoref% Capture \ref in \oldref
  \renewcommand{\autoref}[1]{% Update \ref to use...
    {\reffont\oldautoref{#1}}%
  }%
}
%-------------- reference style ----------------------------------



%------------------ url style ------------------------------------
\AtBeginDocument{%
  \LetLtxMacro\oldurl\url% Capture \ref in \oldref
  \renewcommand{\url}[1]{% Update \ref to use...
    \ul{\oldurl{#1}}
  }% ...\reffont
}
%------------------ url style ------------------------------------



%------------------------ Überschriften -------------------------------
% \addtokomafont{section}{\normalfont\color{Mittelblau}}
% \addtokomafont{subsection}{\color{Grau90}}
% \addtokomafont{subsubsection}{\color{black}}

% Abstand
\RedeclareSectionCommand[
  runin=false,
  beforeskip=.3\baselineskip plus 2pt minus 2pt,
  afterskip=-.1\baselineskip plus 2pt minus 2pt]{section}
  \RedeclareSectionCommand[
  runin=false,
  beforeskip=0.15\baselineskip plus 2pt minus 2pt,
  afterskip=-.2\baselineskip plus 2pt minus 2pt]{subsection}
  \RedeclareSectionCommand[
  runin=false,
  beforeskip=.1\baselineskip plus 2pt minus 2pt,
  afterskip=-.5em plus 2pt minus 2pt]{subsubsection}
  \RedeclareSectionCommand[
  runin=true,
  beforeskip=0pt,
  afterskip=-1em plus 2pt minus 2pt]{paragraph}
  \RedeclareSectionCommand[
  runin=false,
  beforeskip=0pt,
  afterskip=-1em plus 2pt minus 2pt]{subparagraph}
%------------------------ Überschriften -------------------------------



%------------------------ Auflistungen ----------------------------------------------------------------------------
\usepackage{enumitem} % control layout of itemize, enumerate, description - https://ctan.org/pkg/enumitem?lang=de
\setlist{itemsep=-1.5ex plus 2pt minus 2pt}

% Auflistungssymbole - http://tug.ctan.org/tex-archive/macros/latex/contrib/koma-script/doc/scrguide.pdf
%------------------------ Auflistungen ----------------------------------------------------------------------------



%-------------------- Fußnoten ----------------------------------------
\addtokomafont{footnote}{\sffamily}
\addtokomafont{footnotereference}{\sffamily}

\renewcommand{\footnoterule}{%
  \vfill%
  \hskip 8pt plus 2pt minus 1pt
  {%
    \hrule width 0.65\textwidth height 0.65pt
  }%
  \hskip 7pt plus 2pt minus 1pt
  \vspace{4px}
}
%-------------------- Fußnoten ----------------------------------------



%------------------- captions floating environments -----------------------------------------------------------------------
% \addtokomafont{captionlabel}{\dinProRegular}
\usepackage{caption} % Customising captions in floating environments - https://ctan.org/pkg/caption
\captionsetup{skip=\bigskipamount, textfont=sf, font=small}
%------------------- captions floating environments -----------------------------------------------------------------------



%---------------------- figures -------------------------------------------------------------------------

\makeatletter

\newcommand\fs@tbruled{%
  \def\@fs@cfont{\normalfont}%
  \let\@fs@capt\floatc@plain
  \def\@fs@pre{{\color{Mittelblau}\hrule height .5pt \kern2pt} \vspace{\bigskipamount}}%
  \def\@fs@post{\vspace{\bigskipamount} {\color{Mittelblau} \kern2pt\hrule height 1pt \relax}}%
  \def\@fs@mid{\vspace{\abovecaptionskip}}%
  \let\@fs@iftopcapt\iffalse
}

% remove space above wrap figure
\newcommand\fs@wrap{
  \def\@fs@cfont{\rmfamily}\let\@fs@capt\floatc@plain
  \def\@fs@pre{\vskip -\intextsep}
  \def\@fs@post{}%
  \def\@fs@mid{\vspace\abovecaptionskip\relax}%
  \let\@fs@iftopcapt\iffalse
}

\makeatother

\floatstyle{tbruled} % apply style

% change styles for the standard float types figure and table
\restylefloat{figure} 

% wrapfig patch
\BeforeBeginEnvironment{wrapfigure}{\floatstyle{wrap}\restylefloat{figure}}
\AfterEndEnvironment{wrapfigure}{\floatstyle{tbruled}\restylefloat{figure}}


%---------------------- figures -------------------------------------------------------------------------



%--------------------- underline ----------------------------------------------
% new command \ul
\usepackage{lua-ul} % Underlining for LuaLaTEX - https://ctan.org/pkg/lua-ul
\newunderlinetype\beginUl{%
  \leaders\vrule height -.3ex depth .38ex
}
\NewDocumentCommand\ul{m}{{\beginUl#1}}
%--------------------- underline ----------------------------------------------



%--------------- pdf bookmarks -----------------------------------
\usepackage{hyperref} % cross-referencing support (hypertext) - https://ctan.org/pkg/hyperref
\hypersetup{
  pdfencoding=unicode, % Umlaute in pdfbookmarks
  bookmarksopen,
  bookmarksnumbered,
}
%--------------- pdf bookmarks -----------------------------------

\usepackage{scrhack} % Patcht Fremdpakete für Zusammenarbeit mit Koma - https://komascript.de/~mkohm/scrguide.pdf

%
%
%
%
\newcommand{\email}[1]{\href{mailto:#1}{\color{Anthrazitgrau}\sffamily\textless#1\textgreater}}


\makeatletter
\newcommand*{\shifttext}[2]{%
  \settowidth{\@tempdima}{#2}%
  \makebox[\@tempdima]{\hspace*{#1}#2}%
}
\makeatother


\newcommand{\directquote}[1]{%
  \enquote{\textit{#1}}%
}



%----------- Vollmer cite commands ----------------------------------------------
% Meine speziellen \cite-Kommandos:
% Ausgabe des Untertitels (SUBTITLE) Feldes einer Referenz
\DeclareCiteCommand{\citesubtitle}
  {\boolfalse{citetracker}%
   \boolfalse{pagetracker}%
   \usebibmacro{prenote}}
  {\indexfield{indextitle}%
   \printfield[citetitle]{subtitle}}
  {\multicitedelim}
  {\usebibmacro{postnote}}

% Ausgabe des Titels (TITLE) Feldes und der Referenz
\DeclareCiteCommand{\citetitleref}
  {\booltrue{citetracker}%
   \booltrue{pagetracker}%
   \usebibmacro{prenote}}
  {\indexfield{indextitle}%
   \printfield[citetitle]{title} \cite{\thefield{entrykey}}}
  {\multicitedelim}
  {\usebibmacro{postnote}}

% wie citetitleref, nur als Fussnote
\DeclareCiteCommand{\citetitlerefFootnote}
  {}
  {\footnote{\citetitleref{\thefield{entrykey}}}}
  {}
  {}

% Ausgabe der URL im HOWPUBLISHED Feld
% Referenzieren von URL's, Format in der *.bib-Datei
%@MISC{key,
%  AUTHOR	= "....",
%  TITLE	= "Webseite....",
%  HOWPUBLISHED = "\url{http://www.domain.tld}",
%  YEAR		= YYYY,    % Jahr  der Einsichtname, YYYY = Jahreszahl 4 Stellog
%  MONTH	= ABC      % Monat der Einsichtnahme (jan, feb, mar, apr, may, jun, jul, aug, sep, oct, nov. dec
%}
\DeclareCiteCommand{\citeurl}
  {\booltrue{citetracker}%
   \booltrue{pagetracker}%
   \usebibmacro{prenote}}
  {\indexfield{indextitle}%
   \printfield[citetitle]{howpublished}}
  {\multicitedelim}
  {\usebibmacro{postnote}}

% Ausgabe der URL im HOWPUBLISHED Feld und Referenz
\DeclareCiteCommand{\citeurlref}
  {\booltrue{citetracker}%
   \booltrue{pagetracker}%
   \usebibmacro{prenote}}
  {\indexfield{indextitle}%
   \printfield[citetitle]{howpublished} \cite{\thefield{entrykey}}}
  {\multicitedelim}
  {\usebibmacro{postnote}}

% Augabe: Vorname Nachname des Autors
\DeclareCiteCommand{\citefullauthor}
  {\booltrue{citetracker}%
   \booltrue{pagetracker}%
   \usebibmacro{prenote}}
  {\indexfield{indextitle}%
    \textsc{\printnames[byeditor]{author}}}
  {\multicitedelim}
  {\usebibmacro{postnote}}
%----------- Vollmer cite commands ----------------------------------------------

\newcommand{\codeline}[1]{\vskip-1ex\hspace*{3em}\texttt{#1}\vspace{-1ex}}

\newcommand{\urlfootnote}[1]{\footnote{\url{#1}}}

\newcommand\centercaption{%
  \captionsetup{format=plain,labelsep=newline,justification=centering}%
}
% 
% 
% 
% 
% 
% 
% 
% 
% 
% 
% 
% 
% 
% 
% 
% 
% 
% 
% 
% 

% 
% 
% 
% 
% 
% 
% 
% 
% 
% 
% 
% 
% 
% 
% 
% 
% 
% 
% 
% 
% 