\section{Vorwort}
Die Dokumentation orientiert sich an der Reihenfolge der Aufgaben im Bewertungsschema.
Alle Testprogramme funktionieren und die Hardwareansteuerung wurde umgesetzt.



\section{Simulatoren}
Entwickelt wurde ein Simulator für den PIC16F84.
Also eine Software, die das Verhalten dieses Mikrocontrollers nachbildet,
um PIC16F84 Assemblerprogramme auf einem x86 Computer in einer grafischen Oberfläche Ausführen zu können.
Hierbei wurde nicht das genaue Verhalten implementiert, mit allen internen Zuständen,
sondern das Ergebnis nach außen.

\paragraph{Vor- \& Nachteile}



\section{Funktionsweise}

\subsection{Gewählte Technologien}
Die Software ist als Webanwendung umgesetzt.
Es gibt also zwei Teile, in die die Anwendung unterteilt werden kann:
Front- und Back-End.
Das Back-End ist in der Programmiersprache C\# geschrieben, mit der \char`\~90~\% des Codes geschrieben wurde,
die Logik im Front-End in JavaScript.
Zum Beschreiben der grafischen Oberfläche wurde HTML verwendet.


\subsection{Einlesen von Programmen}

\subsubsection{Parser}
Das Einlesen der LST-Dateien funktioniert über einen Parser im Front-End,
der mit dem Parsergenerator Tree-sitter\urlfootnote{https://tree-sitter.github.io/tree-sitter/} erstellt wurde.
Tree-sitter generiert aus einer LR(1) Grammatik C-Code für einen Parser,
der zu WebAssembly compiliert wird, ein Bytecode-Standard der W3C zum Ausführen von Programmen innerhalb des Webbrowsers.
Die Anbindung an die JS-API des Browsers funktioniert über eine 
Bibliothek\urlfootnote{https://github.com/tree-sitter/tree-sitter/tree/master/lib/binding_web} des Tree-sitter Projekts.
Das stellt den Grund dafür, dass dieser Teil der Logik im Front-End ist. 

\js[Auszug der Grammatik zum Einlesen der LST-Dateien (\texttt{tree-sitter-pic/grammar.js}).]{./listings/grammatik.js}


\subsubsection{Befehlsdecodierung}
Die Befehlsdecodierung funktioniert über ein Interpretieren der geparsten Opcodes Strings als Zahlen.
Die Zahlen werden wieder in Strings der Zahlen in Binärdarstellung umgewandelt.


\subsection{Grafische Oberfläche}

\subsubsection{Bedienung}



\section{Zusammenfassung}

\subsection{Ergebnis}


\subsection{Persönliches Fazit (Valerio Cocco)}

